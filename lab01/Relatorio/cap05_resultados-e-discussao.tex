% ----------------------------------------------------------------------
% CAPÍTULO 5 - RESULTADOS E DISCUSSÃO
% ----------------------------------------------------------------------

Nesse capítulo são apresentados os resultados e discussões da aplicação
da MANOVA aos dados apresentados no \autoref{cap:conjuntos-de-dados}.
Para ambos os conjuntos de dados, são expostos neste capítulo a
especificação do modelo ajustado, a análise de resíduo e principais
discussões.

\section{Análise dos Processos Movidos contra Grandes Litigantes}

Para ajuste do modelo foram consideradas todas as 3 respostas do tipo
binárias apresentadas na \autoref{sec:dados-processos} e considerou-se
um modelo com efeitos fixos dispostos de forma aditiva, além de
considerar independência entre as observações. Portanto, os preditores
lineares são descritos da forma

...

\begin{knitrout}
\definecolor{shadecolor}{rgb}{0.969, 0.969, 0.969}\color{fgcolor}\begin{kframe}


{\ttfamily\noindent\bfseries\color{errorcolor}{\#\# Error in ggplot(data = res\_pred, aes(x = residuos)): não foi possível encontrar a função "{}ggplot"{}}}\end{kframe}
\end{knitrout}

\section{Análise Comportamental de Ovelhas Submetidas à Intervenção
Humana}

Para análise do experimento foram consideradas todas as 8 respostas apresentadas \autoref{sec:dados-ovelhas}. Foi ajustado um modelo
multivariado com os efeitos fixos das variáveis sessão experimental, 
momento e linhagem dispostos de forma aditiva. 

Como já mencionado, trata-se de um experimento em que as observações não 
são independentes pois, devido a questões de delineamento, cada animal 
contribui com 9 medidas ao conjunto de dados. Faz-se então a necessidade 
de acomodar as medidas repetidas e as correlações existentes na
especificação do modelo. 

No experimento em questão existem dois conjuntos distintos de variáveis 
resposta: contagens e proporções contínuas. Para as contagens a função
de ligação selecionada foi a logaritmica, para função de variância 
utilizou-se a Poisson-Tweedie. Para as proporções contínuas foi
utilizada a função de ligação logito, com função de variância Binomial. 
Adicionalmente estimou-se o parâmetro de potência para todas as
respostas em análise. Os preditores lineares são descritos da seguinte 
forma:
\begin{equation}
g_{R}(\mu_{R}) = \beta_{0R} + \beta_{1R}\text{Sessao}_{ij} + \beta_{2R}\text{Momento}_{ik} + \beta_{3R}\text{Linhagem}_{il}
\end{equation}

O índice $R$ varia de 1 até 8 e refere-se às variáveis resposta do estudo; o índice $i$ varia de 1 até 180 e diz respeito ao número de observações; $j$ varia de 1 a 3 e indexa as 3 diferentes sessões experimentais; $k$ assume os níveis da variável momento experimental: antes, durante e depois; $l$ assume os níveis da variável linhagem: reativo ou não reativo e, por fim, $g_{1}$ a $g_{4}$ equivalem à função de ligação logaritmica, utilizada nas respostas de contagem e $g_{4}$ a $g_{8}$ representam a função de ligação logito, utilizadas nas respostas que configuram proporções.

Os preditores matriciais, mostados na equação X, comuns a todas as respostas, foram especificados de forma a explicitar que as medidas provenientes do mesmo animal são correlacionadas, bem como as medidas dentro da mesma sessão. 

\begin{equation}
h\left \{ \boldsymbol{\Omega}(\boldsymbol{\tau}) \right \} = \tau_0Z_0 + \tau_1Z_1 + \tau_2Z_2.
\end{equation}

Em que $h(.)$ é a covariance link function; $\tau_0$, $\tau_1$ e $\tau_2$ representam os parâmetros de dispersão; $Z_0$ representa uma matriz identidade de ordem $180 x 180$, $Z_1$ representa uma matriz $180 x 180$ composta por blocos $9 x 9$ de uns na diagonal principal, esta matriz especifica qua as medidas provenientes do mesmo animal são correlacionadas; $Z_2$ representa uma matriz $180 x 180$ composta por blocos $3 x 3$ de uns na diagonal principal indicando que medidas na mesma sessão experimental são correlacionadas. Considerando um único animal, o preditor matricial tem a seguinte forma:

\begin{equation}
h\left \{ \boldsymbol{\Omega}(\boldsymbol{\tau}) \right \} = 
\tau_0 \begin{bmatrix}
1 & 0 & 0 & 0 & 0 & 0 & 0 & 0 & 0\\ 
0 & 1 & 0 & 0 & 0 & 0 & 0 & 0 & 0\\ 
0 & 0 & 1 & 0 & 0 & 0 & 0 & 0 & 0\\ 
0 & 0 & 0 & 1 & 0 & 0 & 0 & 0 & 0\\ 
0 & 0 & 0 & 0 & 1 & 0 & 0 & 0 & 0\\ 
0 & 0 & 0 & 0 & 0 & 1 & 0 & 0 & 0\\ 
0 & 0 & 0 & 0 & 0 & 0 & 1 & 0 & 0\\ 
0 & 0 & 0 & 0 & 0 & 0 & 0 & 1 & 0\\ 
0 & 0 & 0 & 0 & 0 & 0 & 0 & 0 & 1\\ 
\end{bmatrix} + 
\tau_1 \begin{bmatrix}
1 & 1 & 1 & 1 & 1 & 1 & 1 & 1 & 1\\ 
1 & 1 & 1 & 1 & 1 & 1 & 1 & 1 & 1\\ 
1 & 1 & 1 & 1 & 1 & 1 & 1 & 1 & 1\\ 
1 & 1 & 1 & 1 & 1 & 1 & 1 & 1 & 1\\ 
1 & 1 & 1 & 1 & 1 & 1 & 1 & 1 & 1\\ 
1 & 1 & 1 & 1 & 1 & 1 & 1 & 1 & 1\\ 
1 & 1 & 1 & 1 & 1 & 1 & 1 & 1 & 1\\ 
1 & 1 & 1 & 1 & 1 & 1 & 1 & 1 & 1\\ 
1 & 1 & 1 & 1 & 1 & 1 & 1 & 1 & 1\\ 
\end{bmatrix} + 
\tau_2 \begin{bmatrix}
1 & 1 & 1 & 0 & 0 & 0 & 0 & 0 & 0\\ 
1 & 1 & 1 & 0 & 0 & 0 & 0 & 0 & 0\\ 
1 & 1 & 1 & 0 & 0 & 0 & 0 & 0 & 0\\ 
0 & 0 & 0 & 1 & 1 & 1 & 0 & 0 & 0\\ 
0 & 0 & 0 & 1 & 1 & 1 & 0 & 0 & 0\\ 
0 & 0 & 0 & 1 & 1 & 1 & 0 & 0 & 0\\ 
0 & 0 & 0 & 0 & 0 & 0 & 1 & 1 & 1\\ 
0 & 0 & 0 & 0 & 0 & 0 & 1 & 1 & 1\\ 
0 & 0 & 0 & 0 & 0 & 0 & 1 & 1 & 1\\ 
\end{bmatrix}.
\end{equation}

A Tabela \autoref{table:est} mostra as estimativas dos parâmetros de dispersão e potência para ambos os tipos de resposta em estudo.

\begin{table}[H]
\centering
\caption{Estimativas dos parâmetros de dispersão e potência.}
\begin{tabular}{lcccc}
\bottomrule
               & $\phi_{1}$ & $\phi_{2}$  & $\phi_{3}$ & $p$ \\
\bottomrule
  Resposta     & $Est. (ep)$ & $Est. (ep)$  & $Est. (ep)$ & $Est. (ep)$ \\
\bottomrule
  ncorpo & 2.34 (0.69) &  0.42 (0.20) &  -0.22 (0.22) &  1.28  (0.26) \\
  ncabeca & 1.62 (0.38) &  0.79 (0.32) &  -0.11 (0.17) &  1.11  (0.17) \\
  norelha & 1.75 (0.88) &  0.23 (0.20) &  0.14  (0.21) &  1.42  (0.28) \\
  nolho & 1.67 (3.55) &  0.16 (0.38) &  0.18  (0.45) &  1.21  (0.77) \\
  resporelha2 & 0.75 (0.14) & -0.02 (0.03) & 0.07 (0.07) & 1.20 (0.04) \\
  respcauda & 0.59 (0.32) &  -0.08 (0.06) &  0.38 (0.18) &  1.29 (0.11) \\
  respolho & 1.02 (0.48) &  -0.05 (0.08) &  0.40 (0.26) &  1.54 (0.31) \\
  resporlev & 0.06 (0.09) &  0.001 (0.01) &  0.01 (0.02) & -0.27 (1.01) \\
 \bottomrule
\end{tabular}
\label{table:est}
\end{table}

Os valores observados dos parâmetros mostram que o efeito das medidas repetidas por animal é mais acentuado nas respostas de contagem, para as respostas de proporção o parâmetro de dispersão associado ao efeito das medidas repetidas é próximo de zero para todas as respostas, indicando pouco efeito das medidas repetidas. Quanto ao parâmetro de dispersão que mensura o efeito das medidas repetidas dentro de uma mesma sessão experimental, nota-se que houve poucos casos em que o parâmetro estimado é próximo de zero, indicando que há efeito das medidas repetidas nas sessões experimentais.

As estimativas dos parâmetros de potência mostram que as respostas ncorpo, nacabeca e nolho seguem distribuição próxima à Neyman Tipo A, já a resposta norelha segue distribuição próxima à Pólya–Aeppli. Como os parâmetros de potência estão próximos de 1 para as respostas reporelha2, repcalda e respolho há um indício de adequação à distribuição beta para estas respostas, o que não ocorre para a respostas resporlev, contudo a proximidade de 0 do parâmetro de potência para esta resposta indica que, condicional às covariáveis, esta variável tem distribuição simétrica.

\begin{itemize}
  \item $\phi_{i}$ testa a significância do preditor matricial
  \item $p$ para contagens indica qual distribuição se adequa; para proporções, valores próximos de 1 indicam adequação à beta.
\end{itemize}

A Figura \autoref{fig:cor_resul} mostra o correlograma das estimativas da correlação resultantes do modelo. Nota-se que, como esperado, as estimativas das correlações estimadas pelo modelo são menores que as correlações de Pearson mostradas na análise exploraória; tal redução deve-se ao fato de que parte da correlação entre as respostas é explicada pelas variáveis explicativas coletadas no estudo.

\begin{knitrout}
\definecolor{shadecolor}{rgb}{0.969, 0.969, 0.969}\color{fgcolor}\begin{figure}[H]
\includegraphics[width=\maxwidth]{figure/cor_resul-1} \caption[Correlações estimadas pelo modelo]{Correlações estimadas pelo modelo.}\label{fig:cor_resul}
\end{figure}


\end{knitrout}

Com o propósito de verificar a qualidade do ajuste do modelo, investigou-se o comportamento dos erros padrões estimados pelo modelo bem como dos erros robustos e dos erros com vício corrigido. Na Figura \autoref{fig:raz} são mostradas as razões entre os erros robustos e de vício corrigido com os erros padrões originais.



\begin{knitrout}
\definecolor{shadecolor}{rgb}{0.969, 0.969, 0.969}\color{fgcolor}\begin{figure}[H]
\includegraphics[width=\maxwidth]{figure/raz-1} \caption[Razão entre os erros]{Razão entre os erros.}\label{fig:raz}
\end{figure}


\end{knitrout}

Nota-se que a razão entre os erros está, na maior parte dos casos, próximas de 1, indicando que que os erros Robusto e de Vício Corrigido são próximos aos erros originais estimados pelo modelo.

Adicionalmente, foi feita a análise de resíduos do modelo. Na Figura \autoref{fig:hist} são exibidos os histogramas dos resíduos de Pearson por 
resposta.

\begin{knitrout}
\definecolor{shadecolor}{rgb}{0.969, 0.969, 0.969}\color{fgcolor}\begin{figure}[H]
\includegraphics[width=\maxwidth]{figure/hist-1} \caption[Histograma dos resíduos]{Histograma dos resíduos.}\label{fig:hist}
\end{figure}


\end{knitrout}

O comportamento dos resíduos evidencia uma distribuição aproximadamente simétrica em torno de zero com variabilidade constante, entre -2 e 2 com poucos valores atípicos. De forma geral, os resultados apontam para ajuste satisfatório do modelo.

Por fim, foi realaziada a Análise de Variância Multivariada com o objetivo de testar a nulidade simultânea de todos os parâmetros estimados no modelo. O resultado da MANOVA é sumarizado na Tabela \autoref{table:manova}.

\begin{table}[H]
\caption{MANOVA}
\centering
\begin{tabular}{rlrrrr}
  \hline
 & Efeitos & GL & Hotelling Lawley & Qui-quadrado & p\-valor \\ 
  \hline
& Intercepto & 8  & 6.98 & 1257.72 & < 0.01 \\ 
& Sessão     & 16 & 0.18 & 33.27   & < 0.01 \\ 
& Momento    & 16 & 0.78 & 140.57  & < 0.01 \\ 
& Linhagem   & 8  & 0.10 & 19.41   & < 0.01 \\ 
   \hline
\end{tabular}
\label{table:manova}
\end{table}

O resultado do teste de hipótese aponta para a não nulidade simultânea dos parâmetros.
